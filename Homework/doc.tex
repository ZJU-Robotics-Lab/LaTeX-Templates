\documentclass{myart}
\usepackage{amsmath}
\usepackage{amsthm}
\usepackage{colortbl}

\begin{document}
\renewcommand\figurename{Fig}
\thispagestyle{empty}
\newpage

\pagestyle{plain}
\pagenumbering{arabic}

\begin{center}
\huge SOLUTION
\end{center}

\begin{center}
\large Author:Name \qquad Student ID:21*****
\end{center}

%%%%%%%%%%%%%%%%%%%%%%%%%%%%%%
\section{Problem 1}
First, use the bilineartransformation maps:
\begin{equation}
z=\frac{1+s}{1-s}
\end{equation}
\qquad We can get
\begin{equation}
(\frac{15}{4}-K)s^2+\frac{1}{2}s+(K-\frac{1}{4})=0
\end{equation}
\qquad Then, use Routh-Hurwitz criterion
\begin{equation}
\begin{aligned}
s^2& \qquad &\frac{15}{4}-K \qquad &K-\frac{1}{4}\\
s&   \qquad &\frac{1}{2}    \qquad &\\
1&   \qquad &K-\frac{1}{4}  \qquad &
\end{aligned}
\end{equation}
\qquad If the system is stable, it needs to meet the following conditions at the same time
\begin{equation}
\left.
\begin{cases}
\frac{15}{4}>0\\
K-\frac{1}{4}>0
\end{cases}
\right.
\end{equation}
\qquad So, the range of K is
\begin{equation}
\frac{1}{4}<K<\frac{15}{4}
\end{equation}

%%%%%%%%%%%%%%%%%%%%%%%%%%%%%%
\section{Problem 2}
\subsection{(a)}
\begin{equation}
\begin{split}
G(z)&=(1-z^{-1})\cdot Z[\frac{1}{s}\cdot \frac{5}{s(s+1)}] \\
&=(1-z^{-1})\cdot Z[5(\frac{1}{s^2}-\frac{1}{s}+\frac{1}{s+1})] \\
&=\frac{5[(T-1+e^{-T})z+(1-e^{-T}-Te^{-T})]}{(z-1)(z-e^{-T})}
\end{split}
\end{equation}

\begin{equation}
\begin{split}
T(z)&=\frac{G(z)}{1+GH(z)}\\
&=\frac{G(z)}{1+G(z)}\\
&=\frac{5[(T-1+e^{-T})z+(1-e^{-T}-Te^{-T})]}{(z-1)(z-e^{-T})+5[(T-1+e^{-T})z+(1-e^{-T}-Te^{-T})]}
\end{split}
\end{equation}

\subsection{(b)}
Let $a=T-1+e^{-T}$, and $b=1-e^{-T}-Te^{-T}$, then
\begin{equation}
T(z)=\frac{5az+5b}{(z-1)(z-e^{-T})+5az+5b}
\end{equation}
\qquad The characteristic equation of this system is
\begin{equation}
(z-1)(z-e^{-T})+5az+5b=0
\end{equation}
\qquad When T=0.1s, $a=0.1-1+e^{-0.1}\approx 0.0048$, $b=1-e^{-0.1}-0.1e^{-0.1}\approx 0.0047$.\par
The characteristic equation is
\begin{equation}
z^2-1.88z+0.93=0
\end{equation}
\qquad Use Jury's stability criterion
\begin{equation}
\begin{tabular}{p{1cm} p{2cm} p{2cm} p{2cm}}
   &       1&  -1.88& 0.93\\
 -)&    0.93&  -1.88&    1\\
\hline
   &   \cellcolor[rgb]{.99,.3,.3}0.135& -0.132&    0\\
 -)&  -0.132&  0.135&    \\
 \hline
   & \cellcolor[rgb]{.99,.3,.3}0.00593&      0&     \\
\end{tabular}
\end{equation}
\qquad So the system is STABLE when T=0.1s.

\subsection{(c)}
Similarly, when T=1s, $a=1-1+e^{-1}\approx 0.368$, $b=1-e^{-1}-e^{-1}\approx 0.264$.\par
The characteristic equation is
\begin{equation}
z^2+0.47z+1.69=0
\end{equation}
\qquad Use Jury's stability criterion
\begin{equation}
\begin{tabular}{p{1cm} p{2cm} p{2cm} p{2cm}}
   &       1&   0.47& 1.69\\
 -)&    1.69&   0.47&    1\\
\hline
   &   \cellcolor[rgb]{.99,.3,.3}-1.86& -0.324&    0\\
\end{tabular}
\end{equation}
\qquad So the system is UNSTABLE when T=1s.

\subsection{(d)}
The stability of the discrete system is related to the sampling period. If the sampling period is too long, a stable system may not be stable anymore.

%%%%%%%%%%%%%%%%%%%%%%%%%%%%%%
\section{Problem 3}
\subsection{(a)}
Use Matlab to solve this problem.
\lstinputlisting[language=Matlab]{./code/p3.m}
\qquad Root locus:
\midpic{1.eps}{Root Locus}

\newpage
\subsection{(b)}
From Figure 2 and Figure 3, we can get that, the root locus and the real axis has three intersections when $K=47.2$ or $K=1.92$ or $K=9.57\times 10^{-5}$. However, when the two real poles break away from the real axis, $K=9.57\times 10^{-5}$, and split point coordinates is $(0.998,0)$.
\doublepic{2.eps}{$K=47.2$}{3.eps}{$K=1.92$ and $K=9.57\times 10^{-5}$}

\subsection{(c)}
The closed-loop system's characteristic equation is
\begin{equation}
z^4+(K-3.7123)z^3+(10.3614K+5.1644)z^2+(9.7585K-3.195)z+0.8353K+0.7408=0
\end{equation}
\qquad From the root locus we can know that the root locus and the unit circle have only two intersections, and these two intersections are actually a complex conjugate pole pair. So we suppose the intersections of the root locus and the unit circle are$ z_1=e^{j\omega},z_2=e^{-j\omega} $. Substitute it into the original equation (14), we can get the solution:
\begin{equation}
\left.
\begin{cases}
\omega &=-0.000000001295873\\
K &=0.956515005888313
\end{cases}
\right.
\end{equation}
\qquad Thus, the maximum $K$ for stability is $0.956515005888313$.

\newpage
%%%%%%%%%%%%%%%%%%%%%%%%%%%%%%
\section{Problem 4}
Use Matlab to solve this problem.
\lstinputlisting[language=Matlab]{./code/p4.m}
\qquad We can get a bode diagram as shown in Figure 4.
\maxpic{4.eps}{Bode Diagram}
\newpage
I find the following equivalent digital controllers:

Use forward difference:
\begin{equation}
C(z)=\frac{z+9}{z+59}
\end{equation}

Use backward difference:
\begin{equation}
C(z)=\frac{11z-1}{61z-1}
\end{equation}

Use FOH:
\begin{equation}
C(z)=\frac{0.1806z-0.01389}{z-8.757\times 10^{-27}}
\end{equation}

Use tustin's approximation:
\begin{equation}
C(z)=\frac{0.1935z+0.129}{z+0.9355}
\end{equation}

Use tustin's approximation with frequency prewarping. And I choose the critical frequency $W_c = 2.4 rad/s $:
\begin{equation}
C(z)=\frac{0.1794z+0.1488}{z+0.9694}
\end{equation}

Use matched pole-zero method:
\begin{equation}
C(z)=\frac{0.1667z-7.567\times 10^{-6}}{z-8.757\times 10^{-27}}
\end{equation}


\end{document}